\documentclass[12pt]{article}
%%%%%%%%%%%%%%%%%%%%%%%%%%%%%%%%%%%%%%%%%%%%%%%%%%%%%%%%%%%%%
%% Document setup for Syllabus and HW
%% Related to termcal package

% Note that it appears termcal calendars have to start on a Monday!

\usepackage{termcal}
% Examples from  https://sites.google.com/site/mattmastin/teaching/grsc-7700/latex-templates
% and http://tex.stackexchange.com/questions/843/latex-classes-or-styles-for-schedules-and-or-calendars
%%\renewcommand{\arraystretch}{2}


% Few useful commands (our classes always meet either on Monday and Wednesday 
% or on Tuesday and Thursday)

\newcommand{\MWClass}{%
\calday[Monday]{\classday} % Monday
\skipday % Tuesday (no class)
\calday[Wednesday]{\classday} % Wednesday
\skipday % Thursday (no class)
\skipday % Friday 
\skipday\skipday % weekend (no class)
}
\newcommand{\MWFClass}{%
\calday[Monday]{\classday} % Monday
\skipday % Tuesday (no class)
\calday[Wednesday]{\classday} % Wednesday
\skipday % Thursday (no class)
\calday[Friday]{\classday} % Friday 
\skipday\skipday % weekend (no class)
}

\newcommand{\TRClass}{%
\skipday % Monday (no class)
\calday[Tuesday]{\classday} % Tuesday
\skipday % Wednesday (no class)
\calday[Thursday]{\classday} % Thursday
\skipday % Friday 
\skipday\skipday % weekend (no class)
}

\newcommand{\MThFClass}{%
\calday[Monday]{\classday} % Monday
\skipday % Tuesday (no class)
\skipday % Wednesday (no class)
\calday[Thursday]{\classday} % Thursday
\calday[Friday]{\classday} % Friday
\skipday\skipday % weekend (no class)
}

\newcommand{\Holiday}[2]{%
\options{#1}{\noclassday}
\caltext{#1}{#2}
}

\newcommand{\Assigned}[1]{
\textbf{\underline{#1}}
}


% ***********************************************************
% ********************** END HEADER *************************
% ***********************************************************

\usepackage{nth}
%%%%%%%%%%%%%%%%%%%%%%%%%%%%%%%%%%%%%%%%%%%%%%%%%%%%%%%%%%%%%
\pagestyle{empty}
\begin{document}
\begin{center}
{\bf Physics 360/Math 360  \ \ MWF 11:00 - 11:50 PM,  Room:  Dennis 231
}
\end{center}

\setlength{\unitlength}{1in}

\begin{picture}(6,.1) 
\put(0,0) {\line(1,0){6.25}}         
\end{picture}


\vskip.15in
\noindent\textbf{Instructor:} Michael Lerner,  Dennis 221, Phone: 727-LERNERM
\vskip.15in

\vskip.15in
\noindent\textbf{Assignment 6, Due Friday March \nth{7}}
\vskip.15in

\section{Fourier Transforms}
Boas \S7.12 \#1, 4, 10
\\
Related to \#4 and 10, you may do \#13 and 19 for extra credit.

\section{Laplace Transforms}
Boas \S8.8 \#2, 8
\\
Boas \S 8.9 You may do \#4 and 19 for extra credit, assuming they're
familiar from last semester.

\section{Convolutions}
Boas \S 8.10 \#3
\\
Additional (required) problem: Pick two functions of your own choice,
calculate the convolution, graph both functions and the convolution,
explain whether it makes sense graphically.

\section{Delta functions}
Boas \S 8.11 \#20 (only need to verify 11.19 parts d and e), 21

\section{Green's Functions}

Read Boas \S 8.12 example \#2 (p. 463), do problem \#11
\end{document}
