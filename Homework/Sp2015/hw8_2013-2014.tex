\documentclass[12pt]{article}
%%%%%%%%%%%%%%%%%%%%%%%%%%%%%%%%%%%%%%%%%%%%%%%%%%%%%%%%%%%%%
%% Document setup for Syllabus and HW
%% Related to termcal package

% Note that it appears termcal calendars have to start on a Monday!

\usepackage{termcal}
% Examples from  https://sites.google.com/site/mattmastin/teaching/grsc-7700/latex-templates
% and http://tex.stackexchange.com/questions/843/latex-classes-or-styles-for-schedules-and-or-calendars
%%\renewcommand{\arraystretch}{2}


% Few useful commands (our classes always meet either on Monday and Wednesday 
% or on Tuesday and Thursday)

\newcommand{\MWClass}{%
\calday[Monday]{\classday} % Monday
\skipday % Tuesday (no class)
\calday[Wednesday]{\classday} % Wednesday
\skipday % Thursday (no class)
\skipday % Friday 
\skipday\skipday % weekend (no class)
}
\newcommand{\MWFClass}{%
\calday[Monday]{\classday} % Monday
\skipday % Tuesday (no class)
\calday[Wednesday]{\classday} % Wednesday
\skipday % Thursday (no class)
\calday[Friday]{\classday} % Friday 
\skipday\skipday % weekend (no class)
}

\newcommand{\TRClass}{%
\skipday % Monday (no class)
\calday[Tuesday]{\classday} % Tuesday
\skipday % Wednesday (no class)
\calday[Thursday]{\classday} % Thursday
\skipday % Friday 
\skipday\skipday % weekend (no class)
}

\newcommand{\MThFClass}{%
\calday[Monday]{\classday} % Monday
\skipday % Tuesday (no class)
\skipday % Wednesday (no class)
\calday[Thursday]{\classday} % Thursday
\calday[Friday]{\classday} % Friday
\skipday\skipday % weekend (no class)
}

\newcommand{\Holiday}[2]{%
\options{#1}{\noclassday}
\caltext{#1}{#2}
}

\newcommand{\Assigned}[1]{
\textbf{\underline{#1}}
}


% ***********************************************************
% ********************** END HEADER *************************
% ***********************************************************

\usepackage{nth}
%%%%%%%%%%%%%%%%%%%%%%%%%%%%%%%%%%%%%%%%%%%%%%%%%%%%%%%%%%%%%
\pagestyle{empty}
\begin{document}
\begin{center}
{\bf Physics 360/Math 360
}
\end{center}

\setlength{\unitlength}{1in}

\begin{picture}(6,.1) 
\put(0,0) {\line(1,0){6.25}}         
\end{picture}


\vskip.15in
\noindent\textbf{Instructor:} Michael Lerner,  Dennis 221, Phone: 727-LERNERM
\vskip.15in

\vskip.15in
\noindent\textbf{Assignment 8, Due Friday March \nth{29}, end of day}
\vskip.15in

\section{Vector calculus}
\subsection{Line Integrals}
Boas \S 6.8 (Line integrals) \#
1, 6, 

Extra credit: \# 8, 15

\subsection{Green's Theorem}
Problems from \S 6.9 (Green's Theorem) \# 2, 3, 

Extra credit: \# 6

\subsection{Divergence and Divergence Theorem}
Problems from  \S 6.10 \# 1

Extra credit: \# 12

\subsection{Stokes' Theorem}
Problems from \S 6.11 \# 1, 2

Extra credit: \# 16

\subsection{Div, grad, curl}
You're covering a lot of this in your math class, so we won't do a ton
of problems here.
\subsubsection{Fields with zero and non-zero divergence and curl}
Write down a vector field with the following characteristics (a
different field for each item in the list). In each case, you are not
allowed to re-use one of the fields from class. In each case, it's
acceptable if the listed conditions are not met for all point, but you
must then tell me at least one point where the conditions are met
(e.g. ``the divergence is zero at the origin, but positive at
$(2,2)$'')
\begin{itemize}
\item non-zero divergence, zero curl
\item zero divergence, non-zero curl
\item zero divergence, zero curl
\item non-zero divergence, non-zero curl
\end{itemize}

\subsubsection{Murder Mystery}
One often runs into the following problem: given a field $\vV$, find
$\vA$ such that $\vV = \grad{\vF}$ or $\vV=\curl{\vA}$.  As explained
in the section on conservative fields, if you can write $\vV$ as the
gradient of another field $\vF$, we know that $\vV$ is {\em
  conservative}, and we call $\vF$ the {\em potential field} for
$\vV$.  Additionally, is easy to show that, if $vV$ can be written as
$\curl{\vA}$, then $\div{\vV}=0$. To be clear: you can't always write
$\vV$ as the curl of some other field; we're trying to figure
something out about the cases where you {\em can}.

In this problem, you will use the ``murder mystery'' method (used in
several texts, but I think this name comes from the Oregon Bridge
Project folks).

First example: say you're given

\begin{equation}
  \vF = y\uv{i} + (x+2y)\uv{j}
\end{equation}
and you want to know if it's conservative. You decide to check by
determining if it can be written as the gradient of some other field
$U$. Well, what do we know? 

We know that, if $\vU = \grad{F}$,
$U_x=\pd{F}{x}$, so we can immediately guess that $F = xy$ (note:
it could easily be something else, like $xy + \sin(4000e^{\pi y^2})$, but
our strategy will be to start with the easy guess and make it more
complicated if it needs to be later on). 

We also know that $\pd{U}{y} = F_y = (x+2y)$ so we can integrate to
guess that $U = xy + xy^2$. Are these two guesses compatible?
Yes. Taking the partial of $\pd{U}{x}$ gives us the $y$ we were
looking for.

Please read http://www.math.oregonstate.edu/BridgeBook/book/math/mmm
for a bit more in the way of details, complete with some very good
pictures. The actual homework problem here is the one at the bottom of
that page: find a potential function for the vector field

\begin{equation}
  \vG = yz\uv{i} + (xz+z)\uv{j} + (xy+y+2z)\uv{k}
\end{equation}


\end{document}
