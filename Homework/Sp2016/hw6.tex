\documentclass[12pt]{article}
%%%%%%%%%%%%%%%%%%%%%%%%%%%%%%%%%%%%%%%%%%%%%%%%%%%%%%%%%%%%%
%% Document setup for Syllabus and HW
%% Related to termcal package

% Note that it appears termcal calendars have to start on a Monday!

\usepackage{termcal}
% Examples from  https://sites.google.com/site/mattmastin/teaching/grsc-7700/latex-templates
% and http://tex.stackexchange.com/questions/843/latex-classes-or-styles-for-schedules-and-or-calendars
%%\renewcommand{\arraystretch}{2}


% Few useful commands (our classes always meet either on Monday and Wednesday 
% or on Tuesday and Thursday)

\newcommand{\MWClass}{%
\calday[Monday]{\classday} % Monday
\skipday % Tuesday (no class)
\calday[Wednesday]{\classday} % Wednesday
\skipday % Thursday (no class)
\skipday % Friday 
\skipday\skipday % weekend (no class)
}
\newcommand{\MWFClass}{%
\calday[Monday]{\classday} % Monday
\skipday % Tuesday (no class)
\calday[Wednesday]{\classday} % Wednesday
\skipday % Thursday (no class)
\calday[Friday]{\classday} % Friday 
\skipday\skipday % weekend (no class)
}

\newcommand{\TRClass}{%
\skipday % Monday (no class)
\calday[Tuesday]{\classday} % Tuesday
\skipday % Wednesday (no class)
\calday[Thursday]{\classday} % Thursday
\skipday % Friday 
\skipday\skipday % weekend (no class)
}

\newcommand{\MThFClass}{%
\calday[Monday]{\classday} % Monday
\skipday % Tuesday (no class)
\skipday % Wednesday (no class)
\calday[Thursday]{\classday} % Thursday
\calday[Friday]{\classday} % Friday
\skipday\skipday % weekend (no class)
}

\newcommand{\Holiday}[2]{%
\options{#1}{\noclassday}
\caltext{#1}{#2}
}

\newcommand{\Assigned}[1]{
\textbf{\underline{#1}}
}


% ***********************************************************
% ********************** END HEADER *************************
% ***********************************************************

\usepackage{nth}
%%%%%%%%%%%%%%%%%%%%%%%%%%%%%%%%%%%%%%%%%%%%%%%%%%%%%%%%%%%%%
\pagestyle{empty}
\begin{document}
\begin{center}

\vskip.15in
\noindent\textbf{Assignment 6, Due Tuesday Feb \nth{12} by 1:00 PM}
\vskip.15in
\end{center}

\section{Bessel functions}
Read Boad \S12.18, The Lengthening Pendulum. Do problem 9 from that
section. Note: I haven't done this problem yet, so it might be messy!

\section{Boas \S7.2 Wave Review}
In class, we did the following problems:

1, 6, 7, 17, 21. 

\noindent Make sure you understood them.

\section{Boas \S7.4 Average Value}
(Boas \S7.4, Problems \#3, 4, 10, 14) These should also be relatively fast.
\subsection{} 
Find the average value of the function on the given interval. You may
use equation 4.8 if it applies. It's well worth your time to make a
quick sketch of the function, as you may be able to quickly see if the
average value is zero.
\begin{equation}
  \sin x + 2\sin 2x + 3\sin 3x \qquad \textrm{on} \quad (0,2\pi)
\end{equation}

\subsection{}
\begin{equation}
  1 - e^{-x} \qquad \textrm{on} \quad (0,1)
\end{equation}

\subsection{}
\begin{equation}
  \cos x \qquad \textrm{on} \quad (0,3\pi)
\end{equation}

\subsection{}
Read through Problem 13. If the result doesn't seem obvious to you,
you may prove it and I'll grade it for extra credit. Use the result of
Problem 13 to evaluate the following integral without doing any calculation.
\begin{equation}
  \int_0^{4\pi/3}\sin^2\left(\frac{3x}{2}\right) dx
\end{equation}

\end{document}
