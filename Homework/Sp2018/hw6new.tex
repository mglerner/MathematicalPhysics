\documentclass[12pt]{article}
%%%%%%%%%%%%%%%%%%%%%%%%%%%%%%%%%%%%%%%%%%%%%%%%%%%%%%%%%%%%%
%% Document setup for Syllabus and HW
%% Related to termcal package

% Note that it appears termcal calendars have to start on a Monday!

\usepackage{termcal}
% Examples from  https://sites.google.com/site/mattmastin/teaching/grsc-7700/latex-templates
% and http://tex.stackexchange.com/questions/843/latex-classes-or-styles-for-schedules-and-or-calendars
%%\renewcommand{\arraystretch}{2}


% Few useful commands (our classes always meet either on Monday and Wednesday 
% or on Tuesday and Thursday)

\newcommand{\MWClass}{%
\calday[Monday]{\classday} % Monday
\skipday % Tuesday (no class)
\calday[Wednesday]{\classday} % Wednesday
\skipday % Thursday (no class)
\skipday % Friday 
\skipday\skipday % weekend (no class)
}
\newcommand{\MWFClass}{%
\calday[Monday]{\classday} % Monday
\skipday % Tuesday (no class)
\calday[Wednesday]{\classday} % Wednesday
\skipday % Thursday (no class)
\calday[Friday]{\classday} % Friday 
\skipday\skipday % weekend (no class)
}

\newcommand{\TRClass}{%
\skipday % Monday (no class)
\calday[Tuesday]{\classday} % Tuesday
\skipday % Wednesday (no class)
\calday[Thursday]{\classday} % Thursday
\skipday % Friday 
\skipday\skipday % weekend (no class)
}

\newcommand{\MThFClass}{%
\calday[Monday]{\classday} % Monday
\skipday % Tuesday (no class)
\skipday % Wednesday (no class)
\calday[Thursday]{\classday} % Thursday
\calday[Friday]{\classday} % Friday
\skipday\skipday % weekend (no class)
}

\newcommand{\Holiday}[2]{%
\options{#1}{\noclassday}
\caltext{#1}{#2}
}

\newcommand{\Assigned}[1]{
\textbf{\underline{#1}}
}


% ***********************************************************
% ********************** END HEADER *************************
% ***********************************************************

\usepackage{nth}
%%%%%%%%%%%%%%%%%%%%%%%%%%%%%%%%%%%%%%%%%%%%%%%%%%%%%%%%%%%%%
\pagestyle{empty}
\begin{document}
\begin{center}

\vskip.15in
\noindent\textbf{Assignment 6, Due Friday by Noon}
\vskip.15in
\end{center}

\section{Convolution}
Pick two functions of your own choice,
calculate the convolution, graph both functions and the convolution,
explain whether it makes sense graphically.

\subsection{Directional Derivative and Gradient}
Boas \S 6.6 \# 1, 2, 5

\subsection{Computational}
It's been a while since we did a computer exercise. My mistake! We can
use python to calculate gradients (and people often do). Google around
a bit and figure out at least one function that might do that.

You might find the numpy builtin function
(\url{https://docs.scipy.org/doc/numpy/reference/generated/numpy.gradient.html}). You
might find another one.

Turn in a Jupyter notebook where you reproduce the example from the
documentation, explain what was calculated, and then do a similar
calculation (e.g. plug in some different values or some different
function), and explain if the result is what you expected.

\textbf{Turn in your Jupyter Notebook on Moodle}

\subsection{Line Integrals -- Due Monday when you return}
Boas \S 6.8 (Line integrals) \#
1, 6, 8

\end{document}
