\documentclass[12pt]{article}
%%%%%%%%%%%%%%%%%%%%%%%%%%%%%%%%%%%%%%%%%%%%%%%%%%%%%%%%%%%%%
%% Document setup for Syllabus and HW
%% Related to termcal package

% Note that it appears termcal calendars have to start on a Monday!

\usepackage{termcal}
% Examples from  https://sites.google.com/site/mattmastin/teaching/grsc-7700/latex-templates
% and http://tex.stackexchange.com/questions/843/latex-classes-or-styles-for-schedules-and-or-calendars
%%\renewcommand{\arraystretch}{2}


% Few useful commands (our classes always meet either on Monday and Wednesday 
% or on Tuesday and Thursday)

\newcommand{\MWClass}{%
\calday[Monday]{\classday} % Monday
\skipday % Tuesday (no class)
\calday[Wednesday]{\classday} % Wednesday
\skipday % Thursday (no class)
\skipday % Friday 
\skipday\skipday % weekend (no class)
}
\newcommand{\MWFClass}{%
\calday[Monday]{\classday} % Monday
\skipday % Tuesday (no class)
\calday[Wednesday]{\classday} % Wednesday
\skipday % Thursday (no class)
\calday[Friday]{\classday} % Friday 
\skipday\skipday % weekend (no class)
}

\newcommand{\TRClass}{%
\skipday % Monday (no class)
\calday[Tuesday]{\classday} % Tuesday
\skipday % Wednesday (no class)
\calday[Thursday]{\classday} % Thursday
\skipday % Friday 
\skipday\skipday % weekend (no class)
}

\newcommand{\MThFClass}{%
\calday[Monday]{\classday} % Monday
\skipday % Tuesday (no class)
\skipday % Wednesday (no class)
\calday[Thursday]{\classday} % Thursday
\calday[Friday]{\classday} % Friday
\skipday\skipday % weekend (no class)
}

\newcommand{\Holiday}[2]{%
\options{#1}{\noclassday}
\caltext{#1}{#2}
}

\newcommand{\Assigned}[1]{
\textbf{\underline{#1}}
}


% ***********************************************************
% ********************** END HEADER *************************
% ***********************************************************

%%%%%%%%%%%%%%%%%%%%%%%%%%%%%%%%%%%%%%%%%%%%%%%%%%%%%%%%%%%%%

\pagestyle{empty}
\begin{document}
\begin{center}
{\bf Physics 360/Math 360  \ \ MF 12:00 - 12:50 PM,  W 2:30-3:20, Room:  CST 314
}
\end{center}

\setlength{\unitlength}{1in}

\begin{picture}(6,.1) 
\put(0,0) {\line(1,0){6.25}}         
\end{picture}


\vskip.15in
\noindent\textbf{Instructor:} Michael Lerner,  CST 213 221, Phone: 727-LERNERM
\vskip.15in
\makebox[\textwidth]{Name:\enspace\hrulefill}
\vskip.15in

\noindent\textbf{Assignment 2, Due Friday Feb 11, 5:00 PM.}
\vskip.15in


The main goal of this part of the class is to get familiar with the
mechanics of working with series and complex numbers. So, this
assignment includes many short problems. The selection below is a good
one to get you acquainted with the techniques. The starred problems
are the ones you're required to turn in (there are four of them).
\vskip.15in

\noindent \S1.15: 33* \\
\S2.4: 1*,2,3*,11*,12,13 \\

As you know, there's an exam coming up covering complex numbers. I
won't cover logs on the exam, but I will cover things like complex
powers, geometric representations, and trig functions. So, the
following (short) problems are due.

\noindent \S2.5 51\\
\S2.7 9 \\
\S2.9 1, 5, 12 \\
\S 2.10 16, 17, 20\\


Additionally, we'll be using iPython Notebooks for several assignments
in the class. Most of you seem comfortable with what we've done in
class.

\begin{itemize}
  \item Figure out how to use Python on your computer. That either
        means using Google CoLab, or installing the Anaconda Python
        Distribution. If you have trouble with this, let me know. I'm
        quite good at getting it installed.
  \item Use a \texttt{for} loop to evaluate the first 4 partial sums
    of one of the series from the previous problems. \textbf{I will
      set aside time in class on Tuesday for answering questions
      related to this, so please do this sooner rather than later!}
\end{itemize}

\end{document}
