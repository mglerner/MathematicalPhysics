\documentclass[12pt]{article}
%%%%%%%%%%%%%%%%%%%%%%%%%%%%%%%%%%%%%%%%%%%%%%%%%%%%%%%%%%%%%
%% Document setup for Syllabus and HW
%% Related to termcal package

% Note that it appears termcal calendars have to start on a Monday!

\usepackage{termcal}
% Examples from  https://sites.google.com/site/mattmastin/teaching/grsc-7700/latex-templates
% and http://tex.stackexchange.com/questions/843/latex-classes-or-styles-for-schedules-and-or-calendars
%%\renewcommand{\arraystretch}{2}


% Few useful commands (our classes always meet either on Monday and Wednesday 
% or on Tuesday and Thursday)

\newcommand{\MWClass}{%
\calday[Monday]{\classday} % Monday
\skipday % Tuesday (no class)
\calday[Wednesday]{\classday} % Wednesday
\skipday % Thursday (no class)
\skipday % Friday 
\skipday\skipday % weekend (no class)
}
\newcommand{\MWFClass}{%
\calday[Monday]{\classday} % Monday
\skipday % Tuesday (no class)
\calday[Wednesday]{\classday} % Wednesday
\skipday % Thursday (no class)
\calday[Friday]{\classday} % Friday 
\skipday\skipday % weekend (no class)
}

\newcommand{\TRClass}{%
\skipday % Monday (no class)
\calday[Tuesday]{\classday} % Tuesday
\skipday % Wednesday (no class)
\calday[Thursday]{\classday} % Thursday
\skipday % Friday 
\skipday\skipday % weekend (no class)
}

\newcommand{\MThFClass}{%
\calday[Monday]{\classday} % Monday
\skipday % Tuesday (no class)
\skipday % Wednesday (no class)
\calday[Thursday]{\classday} % Thursday
\calday[Friday]{\classday} % Friday
\skipday\skipday % weekend (no class)
}

\newcommand{\Holiday}[2]{%
\options{#1}{\noclassday}
\caltext{#1}{#2}
}

\newcommand{\Assigned}[1]{
\textbf{\underline{#1}}
}


% ***********************************************************
% ********************** END HEADER *************************
% ***********************************************************

%%%%%%%%%%%%%%%%%%%%%%%%%%%%%%%%%%%%%%%%%%%%%%%%%%%%%%%%%%%%%

\pagestyle{empty}
\begin{document}
\begin{center}
{\bf Physics 360/Math 360  \ \ MWF 11:00 - 11:50 PM, Room:  CST 225
}
\end{center}

\setlength{\unitlength}{1in}

\begin{picture}(6,.1) 
\put(0,0) {\line(1,0){6.25}}         
\end{picture}


\vskip.15in
\noindent\textbf{Instructor:} Michael Lerner,  CST 213, Phone: 765-983-1784
\vskip.15in

\noindent\textbf{Assignment 3, Due Thursday Feb 24, end of day.}
\vskip.15in

% Last one ended with 2.10.20

\noindent{\bf Actually due early}

The first part is actually due on Tuesday: we started working on a
problem in class, where you were solving a differential equation via
the table method. For Tuesday, get as far as you can on that
problem. Please feel \textit{encouraged} to post questions on Piazza!
This is meant to be a group problem.

\noindent {\bf Due on the real due date}
\\

\noindent \textbf{Table problem:} Using a table like the one in Boas \S 12.2, solve problem 12.1.1

\noindent{\bf Legendre Problems:}
\\

Next set of problems: In this assignment, as in many of the future
assignments, we will investigate a particular topic in depth, rather
than solving several separate problems. This week, we focus on the
Legendre polynomials, which have broad applicability in mathematical
physics, especially in the modeling of spherically symmetric systems.

The text of the following problems is taken (with some small changes)
from Boyce and DiPrima, Chapter 5, section 3.

The following problems deal with the Legendre equation:


\begin{equation}
  \label{L}
  (1-x^2)y'' - 2xy' + \alpha(\alpha+1)y = 0
\end{equation}

Following the convention of choosing a fundamental set of solutions such that

\begin{align*}
  y_1(x) &= 1 + b_2(x-x_0)^2 + ... \\
  y_2(x) &= (x-x_0) + c_2 (x-x_0)^3 + ... \\
  b_2+c_2 &= a_2
\end{align*}

(Note that these series have already included the fact that one will
be even and one will be odd, a fact that you'll show below.)

Two solutions of the Legendre equation for $|x| < 1$ are

\begin{align*}
  y_1(x) &= 1 - \frac{\alpha(\alpha+1)}{2!}x^2 + \frac{\alpha(\alpha-2)(\alpha+1)(\alpha+3)}{4!}x^4 \\
   &+ \sum_{m=3}^{\infty}(-1)^m\frac{\alpha\cdot\cdot\cdot(\alpha-2m+2)(\alpha+1)\cdot\cdot\cdot(\alpha+2m-1)}{(2m)!}x^{2m}, \\
  y_2(x) &= x - \frac{(\alpha - 1)(\alpha + 2)}{3!}x^3 + \frac{(\alpha-1)(\alpha-3)(\alpha+2)(\alpha+4)}{5!}x^5 \\
  &+ \sum_{m=3}^{\infty}\frac{(\alpha-1)\cdot\cdot\cdot(\alpha-2m+1)(\alpha+2)\cdot\cdot\cdot(\alpha+2m)}{(2m+1)!}x^{2m+1}
\end{align*}

\vskip.1in
\textbf{Legendre Problem 1}
Write out the first 4 terms for $y_1$ and $y_2$.

\vskip.1in
\textbf{Legendre Problem 2} Show that, if $\alpha$ is zero or a positive even
integer $2n$, the series solution $y_1$ reduces to a polynomial of
degree $2n$ containing only even powers of $x$. Find the polynomials
corresponding to $\alpha=0,2,4$. Similarly, show that if $\alpha$ is a
positive odd integer $2n+1$, the series solution $y_2$ reduces to a
polynomial of degree $2n+1$ containing only odd powers of $x$. Find
the polynomials corresponding to $\alpha=1,3,5$. 

(There will be more Legendre problems on the next homework assignment)
\end{document}
