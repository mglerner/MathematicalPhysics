\documentclass[12pt]{article}
%%%%%%%%%%%%%%%%%%%%%%%%%%%%%%%%%%%%%%%%%%%%%%%%%%%%%%%%%%%%%
%% Document setup for Syllabus and HW
%% Related to termcal package

% Note that it appears termcal calendars have to start on a Monday!

\usepackage{termcal}
% Examples from  https://sites.google.com/site/mattmastin/teaching/grsc-7700/latex-templates
% and http://tex.stackexchange.com/questions/843/latex-classes-or-styles-for-schedules-and-or-calendars
%%\renewcommand{\arraystretch}{2}


% Few useful commands (our classes always meet either on Monday and Wednesday 
% or on Tuesday and Thursday)

\newcommand{\MWClass}{%
\calday[Monday]{\classday} % Monday
\skipday % Tuesday (no class)
\calday[Wednesday]{\classday} % Wednesday
\skipday % Thursday (no class)
\skipday % Friday 
\skipday\skipday % weekend (no class)
}
\newcommand{\MWFClass}{%
\calday[Monday]{\classday} % Monday
\skipday % Tuesday (no class)
\calday[Wednesday]{\classday} % Wednesday
\skipday % Thursday (no class)
\calday[Friday]{\classday} % Friday 
\skipday\skipday % weekend (no class)
}

\newcommand{\TRClass}{%
\skipday % Monday (no class)
\calday[Tuesday]{\classday} % Tuesday
\skipday % Wednesday (no class)
\calday[Thursday]{\classday} % Thursday
\skipday % Friday 
\skipday\skipday % weekend (no class)
}

\newcommand{\MThFClass}{%
\calday[Monday]{\classday} % Monday
\skipday % Tuesday (no class)
\skipday % Wednesday (no class)
\calday[Thursday]{\classday} % Thursday
\calday[Friday]{\classday} % Friday
\skipday\skipday % weekend (no class)
}

\newcommand{\Holiday}[2]{%
\options{#1}{\noclassday}
\caltext{#1}{#2}
}

\newcommand{\Assigned}[1]{
\textbf{\underline{#1}}
}


% ***********************************************************
% ********************** END HEADER *************************
% ***********************************************************

%%%%%%%%%%%%%%%%%%%%%%%%%%%%%%%%%%%%%%%%%%%%%%%%%%%%%%%%%%%%%

\pagestyle{empty}
\begin{document}
\begin{center}
{\bf Physics 360/Math 360  \ \ MWF 11:00 - 11:50 PM, Room:  CST 225
}
\end{center}

\setlength{\unitlength}{1in}

\begin{picture}(6,.1) 
\put(0,0) {\line(1,0){6.25}}         
\end{picture}


\vskip.15in
\noindent\textbf{Instructor:} Michael Lerner,  CST 213, Phone: 765-983-1784
\vskip.15in

\noindent\textbf{Assignment 3, Due as specified}
\vskip.15in

\noindent\textbf{For Wednesday Jan 31}

As you know, there's an exam coming up covering complex numbers. I
won't cover logs on the exam, but I will cover things like complex
powers, geometric representations, and trig functions. So, the
following (short) problems are due on Wednesday:

\S2.5 51\\
\S2.7 9 \\
\S2.9 1, 5, 12 \\
\S 2.10 16, 17, 20\\

\noindent {\bf For Friday Feb 2}

Using a table like the one in Boas \S 12.2, solve problem 12.1.1

\noindent {\bf Also For Friday Feb 2}
 
In this assignment, as in many of the future assignments, we will
investigate a particular topic in depth, rather than solving several
separate problems. This week, we focus on the Legendre polynomials,
which have broad applicability in mathematical physics, especially in
the modeling of spherically symmetric systems.

The text of the following problems is taken (with some small changes)
from Boyce and DiPrima, Chapter 5, section 3.

The following problems deal with the Legendre equation:


\begin{equation}
  \label{L}
  (1-x^2)y'' - 2xy' + \alpha(\alpha+1)y = 0
\end{equation}

Following the convention of choosing a fundamental set of solutions such that

\begin{align*}
  y_1(x) &= 1 + b_2(x-x_0)^2 + ... \\
  y_2(x) &= (x-x_0) + c_2 (x-x_0)^3 + ... \\
  b_2+c_2 &= a_2
\end{align*}

(Note that these series have already included the fact that one will
be even and one will be odd, a fact that you'll show below.)

Two solutions of the Legendre equation for $|x| < 1$ are

\begin{align*}
  y_1(x) &= 1 - \frac{\alpha(\alpha+1)}{2!}x^2 + \frac{\alpha(\alpha-2)(\alpha+1)(\alpha+3)}{4!}x^4 \\
   &+ \sum_{m=3}^{\infty}(-1)^m\frac{\alpha\cdot\cdot\cdot(\alpha-2m+2)(\alpha+1)\cdot\cdot\cdot(\alpha+2m-1)}{(2m)!}x^{2m}, \\
  y_2(x) &= x - \frac{(\alpha - 1)(\alpha + 2)}{3!}x^3 + \frac{(\alpha-1)(\alpha-3)(\alpha+2)(\alpha+4)}{5!}x^5 \\
  &+ \sum_{m=3}^{\infty}\frac{(\alpha-1)\cdot\cdot\cdot(\alpha-2m+1)(\alpha+2)\cdot\cdot\cdot(\alpha+2m)}{(2m+1)!}x^{2m+1}
\end{align*}

\vskip.1in
\textbf{Problem 1}
Write out the first 4 terms for $y_1$ and $y_2$.

\vskip.1in
\textbf{Problem 2} Show that, if $\alpha$ is zero or a positive even
integer $2n$, the series solution $y_1$ reduces to a polynomial of
degree $2n$ containing only even powers of $x$. Find the polynomials
corresponding to $\alpha=0,2,4$. Similarly, show that if $\alpha$ is a
positive odd integer $2n+1$, the series solution $y_2$ reduces to a
polynomial of degree $2n+1$ containing only odd powers of $x$. Find
the polynomials corresponding to $\alpha=1,3,5$. 

\vskip.1in \textbf{Problem 3} The Legendre polynomial $P_n(x)$ is
defined as the polynomial solution of the Legendre equation with
$\alpha=n$ that also satisfies the condition
$P_n(1)=1$.\\  

\textbf{(a)} Using the results of Problem 2, find the
first five Legendre polynomials, $P_0(x),...,P_5(x)$. \\  

\textbf{(b)}
Plot the graphs of $P_0(x),...,P_5(x)$ in the range for which we've
demonstrated convergence, $|x|\le1$. You can check your answers with
Wolfram Alpha (e.g. go to wolframalpha.com and type 
``plot 0.5*(3x\verb|^|2-1) from -1 to
1'' in the box), but you need to print out solutions here using Python
+ matplotlib. For this, please refer to the in-class notebooks and ask
your instructor/classmates for help early and often!\\

\textbf{(c)} Find the zeros of $P_0(x),...,P_5(x)$.

\vskip.3in
\noindent {\bf For Wednesday Feb 6}
\vskip.1in

\vskip.1in
\textbf{Problem 4} The Legendre polynomials play an important role in
mathematical physics. For example, solving the potential equation
(Laplace's equation) in spherical coordinates, we encounter the
equation 


\begin{equation*}
  \dd{F(\phi)}{\phi} + \cot\phi\fd{F(\phi)}{\phi} + n(n+1)F(\phi) = 0,\qquad 0 < \phi < \pi
\end{equation*}

Show that the change of variables $x=\cos\phi$ leads to the Legendre
equation with $\alpha=n$ for $y=f(x)=F(\cos^{-1}(x))$ 

Hint: you may need to use the fact that


\begin{equation*}
  \sin(\arccos(x)) = \sqrt{1-x^2};\qquad \cot(\arccos(x)) = \frac{x}{\sqrt{1-x^2}}
\end{equation*}

\textbf{Problem 5} Show that the Legendre equation can also be written as

\begin{equation*}
  [(1-x^2)y']' = -\alpha(\alpha+1)y
\end{equation*}

It then follows that

\begin{equation}
  \label{e1}
  [(1-x^2)P_n^{'}(x)]' = -n(n+1)P_n(x)
\end{equation}
and
\begin{equation}
  \label{e2}
  [(1-x^2)P_m^{'}(x)]' = -m(m+1)P_m(x).
\end{equation}
By multiplying \eqref{e1} by $P_m(x)$ and \eqref{e2} by $P_n(x)$,
\textbf{integrating by parts}, and then subtracting one equation from the
other, show that


\begin{equation}
  \label{orth}
  \int_{-1}^{1}P_n(x)P_m(x)dx = 0 \qquad \mathrm{if} \quad n \ne m
\end{equation}

This property \eqref{orth} of the Legendre polynomials is known as the
orthogonality property. If $m=n$, it can be shown that the value of
the integral in \eqref{orth} is $2/(2n+1)$. 

\vskip.1in
Given a polynomial $f$ of degree $n$, it is possible to express $f$ as a linear combination of $P_0,P_1,...,P_n$:

\begin{equation}
  \label{basis}
  f(x) = \sum_{k=0}^n a_kP_k(x)
\end{equation}

Note that, since the $n+1$ polynomials $P_0,...,P_n$ are linearly
independent, and the degree of $P_k$ is $k$, any polynomial of degree
$n$ can be expressed as \eqref{basis}. \\ Using the result of Problem
7, you can show that 


\begin{equation*}
  a_k=\frac{2k+1}{2}\int_{-1}^1f(x)P_k(x)dx
\end{equation*}

\textbf{but you don't have to! You're done!}

\end{document}
