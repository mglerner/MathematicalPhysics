\documentclass[12pt]{article}
%%%%%%%%%%%%%%%%%%%%%%%%%%%%%%%%%%%%%%%%%%%%%%%%%%%%%%%%%%%%%
%% Document setup for Syllabus and HW
%% Related to termcal package

% Note that it appears termcal calendars have to start on a Monday!

\usepackage{termcal}
% Examples from  https://sites.google.com/site/mattmastin/teaching/grsc-7700/latex-templates
% and http://tex.stackexchange.com/questions/843/latex-classes-or-styles-for-schedules-and-or-calendars
%%\renewcommand{\arraystretch}{2}


% Few useful commands (our classes always meet either on Monday and Wednesday 
% or on Tuesday and Thursday)

\newcommand{\MWClass}{%
\calday[Monday]{\classday} % Monday
\skipday % Tuesday (no class)
\calday[Wednesday]{\classday} % Wednesday
\skipday % Thursday (no class)
\skipday % Friday 
\skipday\skipday % weekend (no class)
}
\newcommand{\MWFClass}{%
\calday[Monday]{\classday} % Monday
\skipday % Tuesday (no class)
\calday[Wednesday]{\classday} % Wednesday
\skipday % Thursday (no class)
\calday[Friday]{\classday} % Friday 
\skipday\skipday % weekend (no class)
}

\newcommand{\TRClass}{%
\skipday % Monday (no class)
\calday[Tuesday]{\classday} % Tuesday
\skipday % Wednesday (no class)
\calday[Thursday]{\classday} % Thursday
\skipday % Friday 
\skipday\skipday % weekend (no class)
}

\newcommand{\MThFClass}{%
\calday[Monday]{\classday} % Monday
\skipday % Tuesday (no class)
\skipday % Wednesday (no class)
\calday[Thursday]{\classday} % Thursday
\calday[Friday]{\classday} % Friday
\skipday\skipday % weekend (no class)
}

\newcommand{\Holiday}[2]{%
\options{#1}{\noclassday}
\caltext{#1}{#2}
}

\newcommand{\Assigned}[1]{
\textbf{\underline{#1}}
}


% ***********************************************************
% ********************** END HEADER *************************
% ***********************************************************

\usepackage{nth}
%%%%%%%%%%%%%%%%%%%%%%%%%%%%%%%%%%%%%%%%%%%%%%%%%%%%%%%%%%%%%
\pagestyle{empty}
\begin{document}
\begin{center}
{\bf Physics 360/Math 360  \ \ MWF 11:00 - 11:50 PM, Room:  CST 225
}
\end{center}

\setlength{\unitlength}{1in}

\begin{picture}(6,.1) 
\put(0,0) {\line(1,0){6.25}}         
\end{picture}


\vskip.15in
\noindent\textbf{Instructor:} Michael Lerner,  CST 213, Phone: 765-983-1784
\vskip.15in

\begin{center}

\vskip.15in
\noindent\textbf{Assignment 5, some due March 18, some due March 29}
\vskip.15in
\end{center}

%% Plot the triangle wave
%% do some even/odd thing, like 7.9.5 or something

\section{Boas \S7.2 Wave Review - Due March 18}
Make sure you understand the following problems

1, 6, 7, 17, 21.

(You do not have to turn them in.)

\section{Boas \S7.4 Average Value - Due March 18}
%(Boas \S7.4, Problems \#3, 4, 10, 14) These should be relatively
%fast. Making a quick sketch of the function will make some of
%these trivial. 
For your reference, these are Boas \S7.4 problems \#3, 4, 10 % and 14 
\subsection{} 
Find the average value of the function on the given interval. You may
use equation 4.8 if it applies. It's well worth your time to make a
quick sketch of the function, as you may be able to quickly see
the average value. Especially when it's zero. If you find yourself
spending more than 5 minutes on any one of these, please post to
Piazza asking for hints, and then move on to the next one.

\begin{equation}
  \sin x + 2\sin 2x + 3\sin 3x \qquad \textrm{on} \quad (0,2\pi)
\end{equation}

\subsection{}
\begin{equation}
  1 - e^{-x} \qquad \textrm{on} \quad (0,1)
\end{equation}

\subsection{}
\begin{equation}
  \cos x \qquad \textrm{on} \quad (0,3\pi)
\end{equation}

%%\subsection{}
%%Read through Problem 13. If the result doesn't seem obvious to you,
%%you may prove it and I'll grade it for extra credit. Use the result of
%%Problem 13 to evaluate the following integral without doing any calculation.
%%\begin{equation}
%%  \int_0^{4\pi/3}\sin^2\left(\frac{3x}{2}\right) dx
%%\end{equation}

\section{Boas \S7.5 Fourier Series}

\subsection{ - Due March 18}
Problem \S7.5.9 (also graph the sum of the first four non-zero terms using Python in
addition to solving)

\subsection{ - This, and everything after it, due March 29}
Problem 12.

\subsection{}
For extra credit, you may do problem 13.
%% MGL: this includes hw4new.tex

\section{Boas \S7.9 Even and Odd functions}

The example that starts on page 367 is excellent. It shows expanding a
given function as a Fourier sine series, a Fourier cosine series, and
a Fourier series (that last one is typically taken to mean that you
have both sine and cosine terms, or that you use the complex
exponential version of Fourier series).

\subsection{} Read through that example, and then do
problem \S7.9.15. Please note that Boas gives you the answer so that
you can check your work!


\end{document}
