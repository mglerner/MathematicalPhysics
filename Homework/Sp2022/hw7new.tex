\documentclass[12pt]{article}
%%%%%%%%%%%%%%%%%%%%%%%%%%%%%%%%%%%%%%%%%%%%%%%%%%%%%%%%%%%%%
%% Document setup for Syllabus and HW
%% Related to termcal package

% Note that it appears termcal calendars have to start on a Monday!

\usepackage{termcal}
% Examples from  https://sites.google.com/site/mattmastin/teaching/grsc-7700/latex-templates
% and http://tex.stackexchange.com/questions/843/latex-classes-or-styles-for-schedules-and-or-calendars
%%\renewcommand{\arraystretch}{2}


% Few useful commands (our classes always meet either on Monday and Wednesday 
% or on Tuesday and Thursday)

\newcommand{\MWClass}{%
\calday[Monday]{\classday} % Monday
\skipday % Tuesday (no class)
\calday[Wednesday]{\classday} % Wednesday
\skipday % Thursday (no class)
\skipday % Friday 
\skipday\skipday % weekend (no class)
}
\newcommand{\MWFClass}{%
\calday[Monday]{\classday} % Monday
\skipday % Tuesday (no class)
\calday[Wednesday]{\classday} % Wednesday
\skipday % Thursday (no class)
\calday[Friday]{\classday} % Friday 
\skipday\skipday % weekend (no class)
}

\newcommand{\TRClass}{%
\skipday % Monday (no class)
\calday[Tuesday]{\classday} % Tuesday
\skipday % Wednesday (no class)
\calday[Thursday]{\classday} % Thursday
\skipday % Friday 
\skipday\skipday % weekend (no class)
}

\newcommand{\MThFClass}{%
\calday[Monday]{\classday} % Monday
\skipday % Tuesday (no class)
\skipday % Wednesday (no class)
\calday[Thursday]{\classday} % Thursday
\calday[Friday]{\classday} % Friday
\skipday\skipday % weekend (no class)
}

\newcommand{\Holiday}[2]{%
\options{#1}{\noclassday}
\caltext{#1}{#2}
}

\newcommand{\Assigned}[1]{
\textbf{\underline{#1}}
}


% ***********************************************************
% ********************** END HEADER *************************
% ***********************************************************

\usepackage{nth}
%%%%%%%%%%%%%%%%%%%%%%%%%%%%%%%%%%%%%%%%%%%%%%%%%%%%%%%%%%%%%
\pagestyle{empty}
\begin{document}
\begin{center}

\vskip.15in
\noindent\textbf{Assignment 7, Due Friday by 5:00 PM}
\vskip.15in
\end{center}

\section{The wave equation -- MAKE SURE YOU DO THIS BEFORE CLASS ON FRIDAY}
Argue that the wave equation\footnote{What is the wave equation? Look
  in Boas Ch. 13, or on the internet!} is a reasonable physical model for
waves. You may make this argument for 1D, 2D, or 3D waves. Does it
seem reasonable for both longitudinal and transverse waves? You may
use the book, the internet, or whatever resources you'd like, but make
sure to cite your sources.

\subsection{Conceptual Understanding}
In the style of Fenyman, and including pictures, write out a proof of
either the divergence theorem or Stokes' theorem. You're free to spend
as much time studying Feynman as you like \textit{before} doing this
problem. But, while you're writing it out, you must put away all
references. You can re-do the problem until you've completed it fully
in a ``closed-notes'' fashion.

\subsection{Green's theorem in the plane}
Boas starts out with Green's theorem in the plane. Look at her
section. Explain how one can derive that from what we covered in Feynman.


\end{document}
