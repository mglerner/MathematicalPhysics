\documentclass[addpoints, 12pt]{exam}
%\documentclass[12pt]{article}
%%%%%%%%%%%%%%%%%%%%%%%%%%%%%%%%%%%%%%%%%%%%%%%%%%%%%%%%%%%%%
%% Document setup for Syllabus and HW
%% Related to termcal package

% Note that it appears termcal calendars have to start on a Monday!

\usepackage{termcal}
% Examples from  https://sites.google.com/site/mattmastin/teaching/grsc-7700/latex-templates
% and http://tex.stackexchange.com/questions/843/latex-classes-or-styles-for-schedules-and-or-calendars
%%\renewcommand{\arraystretch}{2}


% Few useful commands (our classes always meet either on Monday and Wednesday 
% or on Tuesday and Thursday)

\newcommand{\MWClass}{%
\calday[Monday]{\classday} % Monday
\skipday % Tuesday (no class)
\calday[Wednesday]{\classday} % Wednesday
\skipday % Thursday (no class)
\skipday % Friday 
\skipday\skipday % weekend (no class)
}
\newcommand{\MWFClass}{%
\calday[Monday]{\classday} % Monday
\skipday % Tuesday (no class)
\calday[Wednesday]{\classday} % Wednesday
\skipday % Thursday (no class)
\calday[Friday]{\classday} % Friday 
\skipday\skipday % weekend (no class)
}

\newcommand{\TRClass}{%
\skipday % Monday (no class)
\calday[Tuesday]{\classday} % Tuesday
\skipday % Wednesday (no class)
\calday[Thursday]{\classday} % Thursday
\skipday % Friday 
\skipday\skipday % weekend (no class)
}

\newcommand{\MThFClass}{%
\calday[Monday]{\classday} % Monday
\skipday % Tuesday (no class)
\skipday % Wednesday (no class)
\calday[Thursday]{\classday} % Thursday
\calday[Friday]{\classday} % Friday
\skipday\skipday % weekend (no class)
}

\newcommand{\Holiday}[2]{%
\options{#1}{\noclassday}
\caltext{#1}{#2}
}

\newcommand{\Assigned}[1]{
\textbf{\underline{#1}}
}


% ***********************************************************
% ********************** END HEADER *************************
% ***********************************************************

%%%%%%%%%%%%%%%%%%%%%%%%%%%%%%%%%%%%%%%%%%%%%%%%%%%%%%%%%%%%%

\pagestyle{empty}
\begin{document}
\begin{center}
{\bf Physics 360/Math 360  \ \ MWF 11:00 - 11:50 PM, Room:  CST 225
}
\end{center}

\setlength{\unitlength}{1in}

\begin{picture}(6,.1) 
\put(0,0) {\line(1,0){6.25}}         
\end{picture}


\vskip.15in
\noindent\textbf{Instructor:} Michael Lerner,  CST 213, Phone: 765-983-1784
\vskip.15in

\noindent\textbf{Assignment 1, Due Monday Jan 30, 5:00PM.}
\vskip.15in

In class, we reviewed series and were introduced to power series. 

\begin{questions}
\question[5] $e^x$
\begin{parts}
  \part Find the series expansion
  \part Find the region of convergence
\end{parts}
\question[5] $\ln(1+x)$
\begin{parts}
  \part Find the series expansion
  \part Find the region of convergence
\end{parts}
\question[5] $(1+x)^p$
\begin{parts}
  \part Find the series expansion
  \part Find the region of convergence
\end{parts}
\question[5]
  What is the small angle approximation? Knowing what you know
about the topics we've covered in this class, how could you justify
the small angle approximation?
\\
\\

The main goal of the next part of the class is to get familiar with
the mechanics of working with series and complex numbers. So, this
assignment (like the next one) includes many short problems. The
selection below is a good one to get you acquainted with the
techniques. The starred problems are the ones you're required to turn
in, but the others are useful to test your knowledge.  \vskip.15in

\question[5] These three: \S1.10: 1*,2*,3,4,5; \S1.12: 1*

\question[5] These three: \S1.13: 5ab, 7ab*, 9ab, 10ab, 20*; \S1.14: 3* 
\end{questions}

\end{document}
