\documentclass[12pt]{article}
%%%%%%%%%%%%%%%%%%%%%%%%%%%%%%%%%%%%%%%%%%%%%%%%%%%%%%%%%%%%%
%% Document setup for Syllabus and HW
%% Related to termcal package

% Note that it appears termcal calendars have to start on a Monday!

\usepackage{termcal}
% Examples from  https://sites.google.com/site/mattmastin/teaching/grsc-7700/latex-templates
% and http://tex.stackexchange.com/questions/843/latex-classes-or-styles-for-schedules-and-or-calendars
%%\renewcommand{\arraystretch}{2}


% Few useful commands (our classes always meet either on Monday and Wednesday 
% or on Tuesday and Thursday)

\newcommand{\MWClass}{%
\calday[Monday]{\classday} % Monday
\skipday % Tuesday (no class)
\calday[Wednesday]{\classday} % Wednesday
\skipday % Thursday (no class)
\skipday % Friday 
\skipday\skipday % weekend (no class)
}
\newcommand{\MWFClass}{%
\calday[Monday]{\classday} % Monday
\skipday % Tuesday (no class)
\calday[Wednesday]{\classday} % Wednesday
\skipday % Thursday (no class)
\calday[Friday]{\classday} % Friday 
\skipday\skipday % weekend (no class)
}

\newcommand{\TRClass}{%
\skipday % Monday (no class)
\calday[Tuesday]{\classday} % Tuesday
\skipday % Wednesday (no class)
\calday[Thursday]{\classday} % Thursday
\skipday % Friday 
\skipday\skipday % weekend (no class)
}

\newcommand{\MThFClass}{%
\calday[Monday]{\classday} % Monday
\skipday % Tuesday (no class)
\skipday % Wednesday (no class)
\calday[Thursday]{\classday} % Thursday
\calday[Friday]{\classday} % Friday
\skipday\skipday % weekend (no class)
}

\newcommand{\Holiday}[2]{%
\options{#1}{\noclassday}
\caltext{#1}{#2}
}

\newcommand{\Assigned}[1]{
\textbf{\underline{#1}}
}


% ***********************************************************
% ********************** END HEADER *************************
% ***********************************************************

%%%%%%%%%%%%%%%%%%%%%%%%%%%%%%%%%%%%%%%%%%%%%%%%%%%%%%%%%%%%%

\pagestyle{empty}
\begin{document}
\begin{center}
{\bf Physics 360/Math 360  \ \ MWF 11:00 - 11:50 PM, Room:  CST 225
}
\end{center}

\setlength{\unitlength}{1in}

\begin{picture}(6,.1) 
\put(0,0) {\line(1,0){6.25}}         
\end{picture}


\vskip.15in
\noindent\textbf{Instructor:} Michael Lerner,  CST 213, Phone: 765-983-1784
\vskip.15in

\noindent\textbf{Assignment 5, Due Friday March 8th}
\vskip.15in

Note that we will reserve some time in class on Friday to finish up
problems that you have not completed on this assignment, since I'm
giving you two assignments quickly.

\section{Boas \S7.8 Other Intervals}

Do problem 7.8.9. Compare it with 7.5.9 from before.

\section{Boas \S7.9 Even and Odd functions}

The example that starts on page 367 is excellent. It shows expanding a
given function as a Fourier sine series, a Fourier cosine series, and
a Fourier series (that last one is typically taken to mean that you
have both sine and cosine terms, or that you use the complex
exponential version of Fourier series).

\subsection{} Read through that example, and then do
problem \S7.9.15. Please note that Boas gives you the answer so that
you can check your work!


Expand $\sin(x)$ in a
$\cos$ series, or $\cos(x)$ in a $\sin$ series.

Also do 7.9.6, 7.9.23, and 7.9.24.

\section{Parseval}

Problem 7.11.5

\section{Fourier Transforms}

%NOTE: If we do not get to Fourier transforms by Wednesday, this
%section becomes part of the next assignment.
%
Boas \S7.12 \#4, 9, 13
\\
You may do \#10 and \#19 for extra credit.



\end{document}
