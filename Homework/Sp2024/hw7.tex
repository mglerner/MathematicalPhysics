\documentclass[12pt]{article}
%%%%%%%%%%%%%%%%%%%%%%%%%%%%%%%%%%%%%%%%%%%%%%%%%%%%%%%%%%%%%
%% Document setup for Syllabus and HW
%% Related to termcal package

% Note that it appears termcal calendars have to start on a Monday!

\usepackage{termcal}
% Examples from  https://sites.google.com/site/mattmastin/teaching/grsc-7700/latex-templates
% and http://tex.stackexchange.com/questions/843/latex-classes-or-styles-for-schedules-and-or-calendars
%%\renewcommand{\arraystretch}{2}


% Few useful commands (our classes always meet either on Monday and Wednesday 
% or on Tuesday and Thursday)

\newcommand{\MWClass}{%
\calday[Monday]{\classday} % Monday
\skipday % Tuesday (no class)
\calday[Wednesday]{\classday} % Wednesday
\skipday % Thursday (no class)
\skipday % Friday 
\skipday\skipday % weekend (no class)
}
\newcommand{\MWFClass}{%
\calday[Monday]{\classday} % Monday
\skipday % Tuesday (no class)
\calday[Wednesday]{\classday} % Wednesday
\skipday % Thursday (no class)
\calday[Friday]{\classday} % Friday 
\skipday\skipday % weekend (no class)
}

\newcommand{\TRClass}{%
\skipday % Monday (no class)
\calday[Tuesday]{\classday} % Tuesday
\skipday % Wednesday (no class)
\calday[Thursday]{\classday} % Thursday
\skipday % Friday 
\skipday\skipday % weekend (no class)
}

\newcommand{\MThFClass}{%
\calday[Monday]{\classday} % Monday
\skipday % Tuesday (no class)
\skipday % Wednesday (no class)
\calday[Thursday]{\classday} % Thursday
\calday[Friday]{\classday} % Friday
\skipday\skipday % weekend (no class)
}

\newcommand{\Holiday}[2]{%
\options{#1}{\noclassday}
\caltext{#1}{#2}
}

\newcommand{\Assigned}[1]{
\textbf{\underline{#1}}
}


% ***********************************************************
% ********************** END HEADER *************************
% ***********************************************************

%%%%%%%%%%%%%%%%%%%%%%%%%%%%%%%%%%%%%%%%%%%%%%%%%%%%%%%%%%%%%

\pagestyle{empty}
\begin{document}
\begin{center}
{\bf Physics 360/Math 360  \ \ MF 12:00 - 12:50 PM,  W 2:30-3:20, Room:  CST 314
}
\end{center}

\setlength{\unitlength}{1in}

\begin{picture}(6,.1) 
\put(0,0) {\line(1,0){6.25}}         
\end{picture}


\vskip.15in
\noindent\textbf{Instructor:} Michael Lerner,  CST 213 221, Phone: 727-LERNERM
\vskip.15in
\makebox[\textwidth]{Name:\enspace\hrulefill}
\vskip.15in

\noindent\textbf{Assignment 7, Due Tuesday April 9th  and Friday April
12th}
\vskip.15in

\section{This part due on Tuesday}

\subsection{Directional Derivative and Gradient}
Boas \S 6.6 \# 1, 2, 5

\subsection{Line Integrals}
Boas \S 6.8 (Line integrals) \#
1, 6, 8

%\subsection{Finishing our in-class problem - part 1} We evaluated a line
%integral around several paths in class. Finish the parts we did not do
%in class. MAKE SURE TO DO THIS BEFORE CLASS ON MONDAY!

%Use the parabola as the bottom
%of a region, and the edges of the rectangle as the left side and top
%fo the region, and explicitly verify Green's Theorem for the field
%discussed in class.

\section{The wave equation}
Argue that the wave equation\footnote{What is the wave equation? Look
  in Boas Ch. 13, or on the internet!} is a reasonable physical model for
waves. You may make this argument for 1D, 2D, or 3D waves. Does it
seem reasonable for both longitudinal and transverse waves? You may
use the book, the internet, or whatever resources you'd like, but make
sure to \textit{cite your sources}.

\section{This part due on Friday}

\subsection{Conceptual Understanding}
In the style of Fenyman, and including pictures, write out a proof of
either the divergence theorem or Stokes' theorem. You're free to spend
as much time studying Feynman as you like \textit{before} doing this
problem. But, while you're writing it out, you must put away all
references. You can re-do the problem until you've completed it fully
in a ``closed-notes'' fashion.

\subsection{Green's theorem in the plane}
Boas starts out with Green's theorem in the plane. Look at her
section. Explain how one can derive that from what we covered in
Feynman.

\subsection{Divergence and Curl}
\textbf{First!} give yourself no more than 10 seconds each to write
down on a piece of paper whether you think the divergence and curl for
each of the remaining parts of the worksheet are zero, positive, or
negative.

\noindent Now, write up your solutions for all six parts of the worksheet we did in class.


\end{document}
