\documentclass[12pt]{article}
%%%%%%%%%%%%%%%%%%%%%%%%%%%%%%%%%%%%%%%%%%%%%%%%%%%%%%%%%%%%%
%% Document setup for Syllabus and HW
%% Related to termcal package

% Note that it appears termcal calendars have to start on a Monday!

\usepackage{termcal}
% Examples from  https://sites.google.com/site/mattmastin/teaching/grsc-7700/latex-templates
% and http://tex.stackexchange.com/questions/843/latex-classes-or-styles-for-schedules-and-or-calendars
%%\renewcommand{\arraystretch}{2}


% Few useful commands (our classes always meet either on Monday and Wednesday 
% or on Tuesday and Thursday)

\newcommand{\MWClass}{%
\calday[Monday]{\classday} % Monday
\skipday % Tuesday (no class)
\calday[Wednesday]{\classday} % Wednesday
\skipday % Thursday (no class)
\skipday % Friday 
\skipday\skipday % weekend (no class)
}
\newcommand{\MWFClass}{%
\calday[Monday]{\classday} % Monday
\skipday % Tuesday (no class)
\calday[Wednesday]{\classday} % Wednesday
\skipday % Thursday (no class)
\calday[Friday]{\classday} % Friday 
\skipday\skipday % weekend (no class)
}

\newcommand{\TRClass}{%
\skipday % Monday (no class)
\calday[Tuesday]{\classday} % Tuesday
\skipday % Wednesday (no class)
\calday[Thursday]{\classday} % Thursday
\skipday % Friday 
\skipday\skipday % weekend (no class)
}

\newcommand{\MThFClass}{%
\calday[Monday]{\classday} % Monday
\skipday % Tuesday (no class)
\skipday % Wednesday (no class)
\calday[Thursday]{\classday} % Thursday
\calday[Friday]{\classday} % Friday
\skipday\skipday % weekend (no class)
}

\newcommand{\Holiday}[2]{%
\options{#1}{\noclassday}
\caltext{#1}{#2}
}

\newcommand{\Assigned}[1]{
\textbf{\underline{#1}}
}


% ***********************************************************
% ********************** END HEADER *************************
% ***********************************************************

%%%%%%%%%%%%%%%%%%%%%%%%%%%%%%%%%%%%%%%%%%%%%%%%%%%%%%%%%%%%%

\pagestyle{empty}
\begin{document}
\begin{center}
{\bf Physics 360/Math 360  \ \ MWF 11:00 - 11:50 PM, Room:  CST 225
}
\end{center}

\setlength{\unitlength}{1in}

\begin{picture}(6,.1) 
\put(0,0) {\line(1,0){6.25}}         
\end{picture}


\vskip.15in
\noindent\textbf{Instructor:} Michael Lerner,  CST 213, Phone: 765-983-1784
\vskip.15in

\noindent\textbf{Assignment 8, Due Monday April 17th}
\vskip.15in

\section{}
\subsection{Boas \S13.1} Boas 13.1.2
\section{Diffusion/Heat Flow; Schrodinger}
\subsection{Boas \S13.2} 13.2.3, 13.2.7
\subsection{Boas \S13.3} 13.3.1; you must also write down the rest of
the answer to Example 1. It's perfectly fine to use the book as a
reference *before* you write you answer, but I want you to write it
out in your final form without looking at the book.
\section{Steady State Temp in a Rectangular Plate}
\subsection{Additional problem} Most of the problems we've been solving involve an infinite number of terms in the solution. With appropriate boundary conditions, this is not required. Solve the semi-infinite rectangular plate problem with one side of the plate held at

\begin{equation}
  T = \sin\left(\frac{-2\pi x}{L}\right) %+ \cos\left(\frac{3\pi x}{L}\right)
\end{equation}

\begin{itemize}
\item  How many terms do you find in your solution, a finite number, or an
infinite number? 

\item  If it's finite, why should that be true conceptually? How could you
change the initial conditions so that there were an infinite number of terms?

\item  If it's infinite, why should that be true conceptually? How could you
change the initial conditions so that there were a finite number of terms?

 \end{itemize}

\end{document}
