\documentclass[12pt]{article}
%%%%%%%%%%%%%%%%%%%%%%%%%%%%%%%%%%%%%%%%%%%%%%%%%%%%%%%%%%%%%
%% Document setup for Syllabus and HW
%% Related to termcal package

% Note that it appears termcal calendars have to start on a Monday!

\usepackage{termcal}
% Examples from  https://sites.google.com/site/mattmastin/teaching/grsc-7700/latex-templates
% and http://tex.stackexchange.com/questions/843/latex-classes-or-styles-for-schedules-and-or-calendars
%%\renewcommand{\arraystretch}{2}


% Few useful commands (our classes always meet either on Monday and Wednesday 
% or on Tuesday and Thursday)

\newcommand{\MWClass}{%
\calday[Monday]{\classday} % Monday
\skipday % Tuesday (no class)
\calday[Wednesday]{\classday} % Wednesday
\skipday % Thursday (no class)
\skipday % Friday 
\skipday\skipday % weekend (no class)
}
\newcommand{\MWFClass}{%
\calday[Monday]{\classday} % Monday
\skipday % Tuesday (no class)
\calday[Wednesday]{\classday} % Wednesday
\skipday % Thursday (no class)
\calday[Friday]{\classday} % Friday 
\skipday\skipday % weekend (no class)
}

\newcommand{\TRClass}{%
\skipday % Monday (no class)
\calday[Tuesday]{\classday} % Tuesday
\skipday % Wednesday (no class)
\calday[Thursday]{\classday} % Thursday
\skipday % Friday 
\skipday\skipday % weekend (no class)
}

\newcommand{\MThFClass}{%
\calday[Monday]{\classday} % Monday
\skipday % Tuesday (no class)
\skipday % Wednesday (no class)
\calday[Thursday]{\classday} % Thursday
\calday[Friday]{\classday} % Friday
\skipday\skipday % weekend (no class)
}

\newcommand{\Holiday}[2]{%
\options{#1}{\noclassday}
\caltext{#1}{#2}
}

\newcommand{\Assigned}[1]{
\textbf{\underline{#1}}
}


% ***********************************************************
% ********************** END HEADER *************************
% ***********************************************************

%% MGL
\usepackage{amsmath} % for \text{}
\usepackage{amsfonts} % For \text{}
\usepackage{amssymb} % For \text{}
\usepackage{amsthm}
\usepackage{bigints}

\newcommand{\CC}{\mathbb{C}}
\newcommand{\N}{\mathbb{N}}
\newcommand{\Ham}{\mathcal{H}}
\newcommand{\sL}{\mathcal{L}}
\newcommand{\sP}{\mathcal{P}}
\newcommand{\sV}{\mathcal{V}}
%%%%%%%%%%%%%%%%%%%%%%%%%%%%%%%%%%%%%%%%%%%%%%%%%%%%%%%%%%%%%
\usepackage[table]{xcolor}
\usepackage{multirow}

\usepackage{hyperref}
\hypersetup{
    colorlinks=true,
    linkcolor=blue,
    filecolor=magenta,      
    urlcolor=blue,
  }

  \usepackage{float}
  \usepackage[super]{nth}
  \usepackage{booktabs}
\restylefloat{table}

\pagestyle{empty}

\renewcommand{\thefootnote}{\fnsymbol{footnote}}
\textwidth=7in
\textheight=9.75in
\topmargin=-1in
\headheight=0in
\headsep=.5in
\hoffset  -.85in
\begin{document}

\begin{center}
{\bf Physics 125: Matter in Motion (with Calculus)
}
\end{center}

\setlength{\unitlength}{1in}

\hrule

\vskip.15in 
\noindent\textbf{Instructor:} Michael Lerner,  CST 213, Email:
\href{lernemi@earlham.edu}{lernemi@earlham.edu},  Phone: 727-LERNERM
%\vskip.15in

% Instructor Notes
% 2018 Moodle (EC only): https://moodle13-18.earlham.edu/course/view.php?id=10018
% Check out the in-class python notebooks from last go-round.

\begin{description}
  \item[Office Hours:] T 10-11, $\Theta$ 9:00-10:00. Either via Zoom or in CST 213. I also have an open-door policy, and you're
        encouraged to stop in to ask questions whenever my door is
        open. That's most of the time.  %I also have an open-door policy, and you're
        % encouraged to stop in to ask questions whenever my door is
        % open. That's most of the time.
  \item[When and where:] Class meetings will be
        TF 1:00-2:20 in CST 225.  If Zoom sessions are needed, they will
        all use: \url{https://us02web.zoom.us/j/88190538447}
  \item[COVID:] Campus-wide coronavirus policies and information: \url{https://earlham.edu/coronavirus}
  \item[Required Materials:] Mathematical Methods in the Physical Sciences, Boas, 3E\\
        Piazza account
        (\url{piazza.com/earlham/spring2022/physmath360})
  \item[Pre- and Co-requisites:] Math 320 (differential equations) and Math 250 (multivariate calculus).
  \item[Strongly-suggested:] Physics 125, Matter in Motion \textbf{Suggested:} Math 310 (Linear Algebra)
  \item[Course Goals:]
        One of the great strengths of a modern
        physics background is that it allows you to address problems in an
        enormous range of fields, from physics itself to economics, biology,
        ecology, computer science, complex systems, physical chemistry,
        geology, and engineering. The primary objective of this course is to
        provide a systematic introduction to mathematical techniques that will
        serve you both in future physics courses and in modeling interesting
        problems in your domain of choice. A significant secondary objective
        is to develop a level of mathematical sophistication that will allow
        you to confidently and competently explore further material on your
        own.

        A significant
        fraction of the results of modern scientific research can only be
        accurately expressed in the language of advanced mathematics. A glance
        at any of the most prestigious scientific journals in your field of
        interest will quickly confirm this. Thus, even experimental scientists
        or those interested primarily in field work need to achieve a certain
        degree of mathematical sophistication if they are to understand the
        conceptual foundation and interpretation of research results in their
        own fields. This course is intended to provide you with a working
        knowledge of those techniques most commonly encountered in science and
        engineering. In the process of learning these techniques, you will
        also acquire experience in the physical interpretation of mathematical
        models of physical problems. \ \\\vspace{-.3in}

        \begin{enumerate}
          \item Students will learn to apply foundational topics of mathematical physics to a wide array of physical situations.
          \item Students will chose a topic not covered in this class, model it with the techniques of mathematical physics, and present their work to the class.
          \item Students will gain experience in the physical interpretation of mathematical models of physical systems.
        \end{enumerate}

        This course contributes to several of Earlham's overall educational goals: \ \\\vspace{-.3in}
        \begin{enumerate}
          \item Critical reading and thoughtful reflection
          \item Understanding the scientific process
          \item Analytic reasoning
          \item Mathematics proficiency
          \item Group learning
        \end{enumerate}
        
\end{description}       

\newpage

\noindent\textbf{Course Calendar}\footnote{ 
\noindent If we get substantially ahead of this syllabus, we can include extra topics like numerical integrators or Noether's theorem.

\noindent This class has, in some iterations, spent a week covering \textbf{infinite series} (Boas sections 1.1-1.8). This year, we will assume that students are sufficiently familiar with those topics as they were covered in Math 280. \textbf{If this is not the case, let me know as soon as possible}.}


\setlength{\arrayrulewidth}{.4mm}
\setlength{\tabcolsep}{8pt}
{\rowcolors{2}{green!80!yellow!50}{green!70!yellow!40}
  \begin{table}[h]
    \footnotesize
\caption{\footnotesize{Course Schedule (HW, Reading, project work due on day listed; exams handed out on day listed)}}
\label{tab:course-calendar}
\begin{tabular}{l|l|p{0.18\linewidth}|p{0.31\linewidth}|l|l}
\textbf{Week} & \textbf{Date} & \textbf{Read} & \textbf{Topics} & \textbf{HW} & \textbf{Other} \\ \hline
1             & Feb. 1        & 1.10-1.14  & Power Series, Taylor Series &  &                   \\ 
1             & Feb. 4       & 1.15, 2.1-2.6 & Accuracy of Approx, $\CC$ & 1 &                   \\ \hline 
2             & Feb. 8        & 2.7-2.10, 2.11-2.13 & Cmplx Power Series, $f(z)$,$z^x$ &  &                  \\ 
2             & Feb. 11        &2.14-2.15, 12.1-12.2 & $x^z$, \textbf{Series Solns} of \nth{2} Order ODEs & 2 &                   \\ \hline 
3             & Feb. 15       & 12.5, 12.7 & Generating functions & & Mini-Midterm                  \\ 
3             & Feb. 18       & 12.8, 12.11 & Orthogonality, Frobenius & 3 &                   \\ \hline 
4             & Feb. 22       & 12.12-12.13, 12.14 & Bessel's Equation &  &                  \\ 
4             & Feb. 25       & \textbf{Early Sem Break} &  & 4 &                 \\ \hline 
5             & Feb. 29      & 7.1-7.4 & \textbf{Fourier Series}  &  &                  \\ 
5             & Mar. 4      & 7.5-7.7 & More Fourier & 5 &                 \\ \hline 
6             & Mar. 8      & 7.8-7.9, & Other Intervals, Even and Odd Fns &  &                  \\ 
6             & Mar. 11        & 7.12, [*]Extra Material &Fourier Transform, [*]Orthogonal Fns & 6 &                 \\ \hline 
%7             & Mar. 15        & 8.8-8.10 & \textbf{Laplace Transform}, Convolutions & & Midterm 1                 \\ 
%7             & Mar. 18        & 8.11 & Dirac Delta Functions & 7 &                 \\ \hline 
%              & Mar. 22        & \textbf{Spring Break} &  &  &                   \\ 
%              & Mar. 25        & \textbf{Spring Break} &  &  &                   \\ \hline
%8             & Mar. 29       & 8.12, 2.16 & Green's functions, optics  &  &                   \\ 
%8             & Apr. 1      & none & catch up & 8 &                   \\ \hline 
7             & Mar. 15        & -7.11 & Finish Fourier Series & &                  \\  
7             & Mar. 18        & 7.12, 8.8-8.9  & Transforms: Fourier and Laplace & 7 &                 \\ \hline 
              & Mar. 22        & \textbf{Spring Break} &  &  &                   \\ 
              & Mar. 25        & \textbf{Spring Break} &  &  &                   \\ \hline
8             & Mar. 29       & 8.10-8.11 & Convolutions, Dirac Delta Functions  &  & Midterm 1                  \\  
8             & Apr. 1      & 8.11-8.12 & Dirac Delta Functions, Green's functions & 8 &                   \\ \hline  
9             & Apr. 5       & 6.1-6.6 & \textbf{Vector Analysis} basic concepts, directional derivative &  &                  \\ 
9             & Apr. 8        &(catch up) & & 9 &                   \\ \hline 
10             & Apr. 12      &6.6-6.8 & gradient, $\nabla$, line integrals &  &  Project Topics                 \\ 
10             & Apr. 15       &6.9, 6.10 & Green's Thm, Divergence Thm & 10 &  Paragraph about project                 \\ \hline 
11             & Apr. 19       &  6.11 & Curl and Stokes Theorem &  &                \\ 
11             & Apr. 22       & \textbf{Project Workday} &  & 11 &                 \\ \hline 
12            & Apr. 26      & 13.1-13.3 & \textbf{PDEs} Laplace, heated plate &  & Project Draft 1                 \\ 
12             & Apr. 29     & \textbf{Project Workday} &  & 12 &   Midterm 2               \\ \hline 
13             & May 3    & 13.4 & Wave Equation, Vibrating String &  & Project Draft 2                 \\ 
13             & May 6       & \textbf{Project Workday} &  & 13 &                 \\ \hline 
14             & May 10       &13.5, 13.6 & Steady state temp in cylinder, circular membrane &  &                  \\ 
14             & May 13       & 13.6, 13.7 & circular membrane, sphere, diffusion & 14 &                 \\ \hline \hline
              & May  19      & Final Exam & 1:30-3:30. Take-home final, in-person presentations.  &                      &                   \\ 
\end{tabular}
\end{table}
% Look at older syllabi to see what has been removed here. In particular, numerical integration and (longer ago) Noether's theorem.

% I kind of can't believe we didn't get to
% 13.8, 13.9 & Poisson's % equation, Integral transform solutions to PDE
% and it breaks my heart.

% \begin{tabular}{p{1cm}|p{2cm}|p{2cm}|p{0.25\linewidth}|p{2cm}|p{1cm}|p{1cm}}

\newpage

\noindent \textbf{Major Topics:} 

\begin{center} \begin{minipage}{5in}
\begin{flushleft}
%Infinite Series \dotfill 3 days \\ % -1
Power Series \\
Computational Methods \\
Complex Numbers \\
Series Solution of Second Order Linear Equations \\
Vector Analysis\\
Fourier Series and Transforms \\
Laplace Transforms \\
Dirac Delta Functions \\
%Numerical Integration \dotfill 3 days \\ % +3
Partial Differential Equations \\
Bessel Functions \\
Numerical integration
\end{flushleft}
\end{minipage}
\end{center}
\vskip.1in
\begin{description}
  \item[Recommended Textbooks] \hfill \\
    %     \textbf{Boyce and DiPrima, Elementary Differential Equations and BVPs}
    %     This was the differential equations book last year. We'll either use
    %     parts of it or refer to your current book.
        \textbf{Boyce and DiPrima, Elementary Differential Equations and BVPs}
        This was the differential equations book last semester. It's an
        excellent reference for PDEs later in the term.\\
        \textbf{The Feynman Lectures on Physics, Volume II}
        This has some very nice explanations of the vector calculus concepts
        that we'll be covering, taken from a physicist's perspective. This
        part of Feynman is also extremely conversational and readable. You can read the whole thing online with beautiful typesetting at
        http://www.feynmanlectures.info/ \\
        \textbf{Gelfand and Fomin, Calculus of Variations}
        The best introduction I know of to calculus of variations. It's an
        \$11 Dover paperback. \\
        \textbf{Emmy Noether's Wonderful Theorem} by Dwight E. Neuenschwander
        will be the basis for our discussion of Noether's Theorem, if we cover
        Noether's theorem. This book
        is really well written, and goes much farther than we'll have time
        for. \\
        \textbf{Kusse and Westwig, Mathematical Physics}
        This text is extremely well written. It doesn't have quite the right
        focus for this particular class. \\ 
        \textbf{Arfken and Weber, Mathematical Methods for Physicists}
        This encyclopedic volume makes for a good reference, but is a bit too
        dry to learn from in this course. \\ 
        \textbf{Schey, div grad curl and all that}
        This is an extremely conversational introduction to/refresher on vector calculus \\

  \item[Communication:] My preferred mode of communication is email. I
        will check my email multiple times per day during the work
        week. If you have sent me an email and have not heard back
        from me within 24 hours (excluding weekends), you should email
        me again or drop by my office to ask your question in
        person. I cannot guarantee that I will regularly check my
        email on evenings or on weekends. 
  
        I expect that you will check your Earlham email at least once per
        day. That is, 24 hours after sending an email to the class, it
        is my expectation that all of you will have read that
        email. Emailed assignments, due dates, policy changes, and
        other course information carry the same weight as information
        contained in the syllabus. 

        Information relevant to the course will also be posted to the Moodle
        page. You are responsible for checking Moodle at the same
        frequency and with the same conditions applied to email
        communication.
        \newpage
        
  \item[Workload Expectations:] The expectation is that you will spend
        2-3 hours outside class for every hour in class. So, for this
        three-credit class, you can expect to spend roughly 6-9 hours
        per week on reading, homework and exams. \textbf{If you find
        yourself spending significantly more (or less) than this,
        please let me know ASAP! That's an instructor-error that I can
        easily fix.}
        
  \item[Contacting me about sensitive issues:] What if you have
        something you want to talk about, but don't feel comfortable
        doing so in a public (or non-anonymous) space? This often
        comes up with issues of racism, sexism, and other kinds of
        discrimination. It can come up with issues that relate to my
        behavior, the behavior of your classmates, or the world in
        general. Please feel {\it encouraged} to use either the
        anonymous forum on Moodle (I can't see who wrote something in
        the anonymous Moodle forum, though I {\it can} see who writes
        anonymous Piazza comments), or to leave a written note under
        my door. If you want to write a note, you may want to type it
        up and print it out so that your handwriting is further
        anonymized. Of course, if you feel comfortable and safe doing
        so, you are also encouraged to talk publicly or
        non-anonymously about these things! 

        If you are not comfortable with any of these options, you are
        encouraged to contact the Vice President for Student Life,
        Bonita Washington-Lacey (\url{mailto:washibo@earlham.edu}).

  \item[Academic Enrichment Center (tutoring, study help):] The
        Academic Enrichment Center (AEC) provides assistance with
        study habits and skills as well as a peer tutoring
        service. The AEC is staffed by trained peer tutors for either
        pre-arranged group tutoring sessions (provided for many math,
        science and social science courses) or one-on-one tutoring
        sessions for other courses. Peer tutoring is a free service
        offered to all Earlham students. While we typically do not
        have tutors assigned to this class, we can definitely find
        one-on-one tutors for you at your request! Please visit
        \url{https://earlham.edu/academics/academic-support-and-special-programs/academic-enrichment-center} 
        for more information.

  \item[Diversity Statement:] I can hardly overstate how much inclusion,
        equity and diversity considerations have shaped my entire thinking
        around teaching and practicing physics. It affects everything from the
        way that physics is framed historically to the way that it is
        typically taught in your classrooms. My main goal is to create an
        environment that tells all of us that we can do physics; that invites
        people into physics who have never been invited, or who have been
        explicitly included. In the American physics context, these issues are
        most commonly framed in terms of gender, race, and the typical
        colonial framing of a physics curriculum. In most of our
        introductory classes, we have a ``Non-Newtonian Physicist''
        assignment that focuses specifically on decolonising an
        introductory physics curriculum. In this class, you can choose
        whether to do this as a separate assignment, or whether to
        integrate into your independent project.

        Physics cannot be divorced from the world in which we study it. I
        expect that we will recognize that in this class, and I expect
        that we will create an inclusive and welcoming
        environment. More specifically, I expect that we will strive
        \textit{not} to create an environment that excludes
        people. This is hard work. We will get things wrong. You will
        get things wrong. \textit{I} will get things wrong. Especially
        when I get things wrong, I encourage you to refer to the
        resources in the previous section of the syllabus (direct
        comments, anonymous Moodle comments, anonymous comments under
        my door, and comments to Bonita).

        \newpage

  \item[Grading Policy and Late Policy:]
        Your grade in this class will be determined by a weighted
        combination of your scores from multiple categories. Within
        each category, each assignment is weighted equally. It is
        extremely hard to catch up on late work in normal times
        (usually things are late because students are busy, and people
        don't tend to get less busy!). This is especially true in a
        pandemic. So, I will drop a roughly a week worth of your
        lowest grades from most categories. If you have to turn things
        in late, my strong suggestion is that you prioritize getting
        your schedule back under control. Feel free to simply
        \textit{not do} things that will not be graded. Especially if
        it helps you keep the rest of your life under control.

        Because things can be even harder to schedule during a
        pandemic, I'm adding a second policy this year: a late pass
        policy that follows my policy from last year:

        All students start the term with 5 free late
        passes. You can spend a late pass to extend the deadline of a
        single assignment to the following class period without any
        penalty to your score. Late passes cannot be applied to
        quizzes, or exams, but will work for
        every other type of assignment. Once you have spent your 5
        free late passes, you can request additional extensions, but
        they come with a penalty to your score. For the 6th late pass,
        I will deduct 10\% of your points on that assignment; for the
        7th late pass, I will deduct 20\% of your points on that
        assignment; and so on.
        
        Late passes are not automatic. You must always email me to let me know
        that you plan to use a late pass on a given assignment. Once
        you declare to me that you are spending a  late pass on an
        assignment, that late pass is spent whether or not you follow
        through and actually turn in the assignment by the modified
        due date.

\setlength{\arrayrulewidth}{1mm}
\setlength{\tabcolsep}{8pt}
%\renewcommand{\arraystretch}{2.5}
{\rowcolors{2}{green!80!yellow!50}{green!70!yellow!40}
\begin{table}[H]
%\caption{Grading Breakdown}
\label{tab:overall-grade-weights-table}
\begin{tabular}{llll}
\hline
\textbf{Assignment}                                              & \textbf{Number} & \textbf{Drop} & \textbf{Weight} \\ \hline
\textbf{Attendance/preparation/participation (including Piazza)} & 27     & 3    & 10     \\
\textbf{Homework }                                         & 14     & 1    & 33     \\
\textbf{Non-Newtonian Physicist }                          & 1      & 0    & 3     \\
\textbf{Mini-midterm (open-book) }                                        & 1      & 0    & 5     \\
\textbf{Two midterms, equally-weighted }                                    & 2      & 0    & 20     \\
\textbf{Final exam }                                              & 1      & 0    & 14     \\
\textbf{Independent Project }                                    & 1      & 0    & 15     \\ \hline 
\end{tabular}
\end{table}
        }

        \newpage
        Final grades will be calculated according to the following
        scale, although individual quiz and exam scores may be curved.

        % Please add the following required packages to your document preamble:
% \usepackage{booktabs}
\setlength{\arrayrulewidth}{1mm}
\setlength{\tabcolsep}{8pt}
%\renewcommand{\arraystretch}{2.5}
{\rowcolors{2}{green!80!yellow!50}{green!70!yellow!40}
\begin{table}[H]
%\caption{Grading Breakdown}
\label{tab:overall-grade-cutoffs-table}
\begin{tabular}{ll}
\hline
 \textbf{Grade} & \textbf{Range}       \\ \hline
A+    & 99.1-100\%  \\
A     & 93.0-99.0\% \\
A-    & 90.0-92.9\% \\
B+    & 87.0-89.9\% \\
B     & 83.0-86.9\% \\
B-    & 80.0-82.3\% \\
C+    & 77.0-79.9\% \\
C     & 73.0-76.9\% \\
C-    & 70.0-72.9\% \\
D+    & 67.0-69.9\% \\
D     & 63.0-66.9\% \\
D-    & 60.0-62.9\% \\
F     & 0.0-59.9\% 
\end{tabular}
\end{table}
        }
        

  \item[Standard Grading Rubric:]
        Several of the problems assigned in this class are
        quite challenging. Others are just rote computation. For the more
        challenging problems, my goal is to have you make the strongest
        possible effort towards \textbf{understanding} the solution. Thus, if
        you cannot fully solve the problem, say whatever you can about the way
        in which a solution would proceed from where you stop; say whatever
        you can about the qualitative behavior of a solution; say whatever you
        can about the physical meaning of the solution. 

        Homework problems will be graded on roughly the same scale as used in
        Physics 125 and 235:
        % Please add the following required packages to your document preamble:
% \usepackage{booktabs}
\setlength{\arrayrulewidth}{1mm}
\setlength{\tabcolsep}{8pt}
%\renewcommand{\arraystretch}{2.5}
{\rowcolors{2}{green!80!yellow!50}{green!70!yellow!40}
\begin{table}[H]
%\caption{Grading Breakdown}
\label{tab:standard-grading-rubric-table}
\begin{tabular}{ll}
\hline
 \textbf{Grade} & \textbf{Explanation}       \\ \hline
  5 & Solution is complete and well-written\\
  4 & Solution is missing minor parts or some important explanations\\
  3 & Solution is missing major parts and/or has few if any explanations\\
  2 & At least one major portion of the problem correct\\
  1 & Very little coherent initial effort was expended\\
  0 & No initial solution was submitted\\
\end{tabular}
\end{table}
        }

  \item[Academic Accommodations:] 
        Students with a documented disability (e.g., physical,
        learning, psychiatric, visual, hearing, etc.) who need to
        arrange reasonable classroom accommodations must request
        accommodation memos from the Academic Enrichment Center(main
        floor of Lilly Library) and contact their  instructors each
        semester. For greater success, students are strongly
        encouraged to visit the Academic Enrichment Center within the
        first two weeks of each semester to begin the process.
        {\small
        \url{https://www.earlham.edu/academic-enrichment-center/disability-services}}


        We at Earlham appreciate the contribution that athletes make
        to the vitality of our community. If you must miss class due
        to a competition, notify me well in advance so we can make
        arrangements for you to make up the work. If you do not notify
        me in advance, there is every chance that I may not be able to
        make special arrangements.

  \item[Attendance/preparation/participation (including Piazza):]
        Especially in a pandemic, attendance is (1) critical (2)
        sometimes hard to sustain. During class, we will cover
        material via lecture, and we will work on problems together and via groups in breakout rooms. This
        problem solving is a critical component of the class, and it is the main reason
        attendance is required. We all know that circumstances
        sometimes make it hard or impossible to attend, so I will drop
        your three lowest attendance and participation grades.

        It is very important that students study the appropriate text
        assignments before coming to class and that they be prepared either to
        ask questions or to discuss questions related to the assigned
        material.  The subject matter of this course is sufficiently complex
        that it is neither desirable nor possible to cover all relevant
        details in class.  The primary value of lectures and class discussions
        in this course is to explain and illustrate points which students find
        difficult or obscure.  It is essential, therefore, that students put
        in the advance work required to identify such points; the instructor
        cannot anticipate all of them.  It is hoped that all of us will work
        together to produce a class atmosphere which is conducive to lively
        and interesting discussions of the material.
        Readings are
        assigned for every class period. By \textbf{midnight before each
        class period, you must comment on Piazza}. Your comment can be
        in one of the three following categories: (1) something
        interesting from the reading, (2) something confusing from the
        reading, (3) an answer to someone else’s question. In order to
        make sure we have a good, constructive discussion atmosphere,
        please note that disrespectful comments will receive no credit
        and may be removed. You can post anonymously; instructors can
        see who made each post, but students cannot.
        \textbf{The piazza comments count for 5\% of your grade.}
        
  \item[Exams:] There will be three midterms and a final. The first
        midterm (the ``mini-midterm'') will be open-book and
        untimed. The other tests will be timed and closed-book. All
        will be self-scheduled take-home exams. The final will be
        cumulative, and we will have a clear discussion and review of topics before
        the exam. We will use the scheduled final exam time for
        project presentations.
        If you have a time conflict, you must reach out to me in
        advance of the date of the final.

  \item[Homework:] This is basically a course in
        problem-solving techniques.  The most common and the most damaging
        mistake a student can make in such a course is to yield to the
        temptation to try and master the material by reading the text and
        listening to the lectures while working a minimum of problems.  While
        reading and listening can give you valuable new ideas, problem-solving
        skills are only efficiently developed by solving as many problems as
        possible.  For most students the understanding and knowledge gained
        from this course will be in direct proportion to the number of
        problems which they successfully complete.  The best approach is to
        maximize the percentage of your study time devoted to a disciplined
        effort to solve problems.
        
        Homework will be assigned each approximately once per week. As with
        all Physics classes, you will find that your understanding is greatly
        improved if you start the assignments early in the week, rather than
        late in the week.
        
        \newpage
  \item[Independent project:] The primary goals of this course are
        involve developing a level of mathematical sophistication that
        will allow you to confidently and competently explore further
        material outside of a lecture setting. So, in order to make
        sure we're walking the walk, you'll each pick either an
        interesting problem to model or an interesting technique to
        learn. You'll write a short paper and present the results to
        the class. You'll be expected to start on this halfway through
        the semester, and we'll discuss it in more detail at that
        point. Examples might include 

        \begin{itemize}
          \item The Fast Fourier Transform (FFT)
          \item Physical representations of Fourier Transforms, including circuits and optics
          \item The discrete Fourier Transform, including application to time series data, consideration of how many useful samples you have in a dataset in Fourier space, etc.
          \item Struve functions
          \item Complex analysis including itegrating around poles, Cauchy
                Integral, Residue theorem (\textbf{this is a hard topic to nail
                down in terms of applications; you must carefully verify your
                proposal with me first! })
          \item Tensors, covariant vs. contravariant, pseudo-vectors, index notation
          \item Black-Scholes Theorem (economics)
          \item Applications of Noether's Theorem to economic systems
          \item Modern techniques such as the immersed-boundary method
          \item Coding and examination of more complex numerical integrators
          \item Green-Kubo relations and the Fluctuation Theorem
          \item Wavelet Transforms
          \item Real world applications of any of the techniques studied in this course
          \item Numerical integration (e.g. write your own RK4 and Velocity 
                Verlet, compare the two for some real application)
          \item Noether's Theorem
          \item The physics of a golf swing
          \item Building a better baseball lineup with Monte Carlo Methods
          \item Biomedical examples of Laplace transforms
          \item Interesting topic of your choice! 
        \end{itemize}

\textbf{Note}: If there is sufficient demand, it is likely possible to include at least one of the above topics in the main body of the course itself.
        
    \item[Academic Integrity:] {\small
        \url{http://www.earlham.edu/curriculum-guide/academic-integrity/}}
        As with all Earlham courses, this course is covered by the
        academic integrity policy. Much of the work in this class is
        group-based, but not all of it. Specific expectations are
        discussed in class and in this syllabus, but: daily problems
        should be attempted individually first. You may then 
        ask for guidance in class, in the forums, and from your
        peers. Quizzes and exams must be done individually.

  \item[Sexual Harassment and Title IX] Federal law, Title IX, and
        Earlham College policy prohibits discrimination, harassment,
        and violence based on sex and gender (including sexual
        harassment, sexual assault, domestic / dating violence,
        stalking, sexual exploitation, and retaliation). If you or
        someone you know has been harassed or assaulted, you can
        receive confidential support from a number of resources on and
        off campus, listed here:
        \url{http://earlham.edu/counseling-services/confidential-resources-for-survivors-of-sexual-violence/}.

        Alleged 
        violations can be reported non-confidentially to the Title IX
        Coordinator, Jenelle Job, at ext. 1346. Reports to law
        enforcement can be made to Earlham Public Safety at ext. 1400,
        or to the Richmond Police Department at 765-983-7247 (for
        non-emergency calls). I will seek to keep information you
        share with me private to the greatest extent possible, but as
        a professor I have mandatory reporting responsibilities
        regarding sexual misconduct and crimes I learn about to help
        make our campus a safer place for all. 
        
\end{description}



\end{document}
